\chapter{开发日记}


\section{2023年10月}


\subsection{10月5日}
这个项目最初只是打算使用简单的判断方法来解决,但想到以后要开发其他编译器,所以先用这个微小项目来练练手。项目将按照编译器开发方法来实现,当然,本项目过于微小,大概只会用到词法分析与语法分析。

Windows系统和Linux系统在文本处理上是有些差异的,其中的换行就不相同。Windows系统中的换行实际上包含了两个字符,即$\backslash r$(回车,0xD)与$\backslash n$(换行,0xA),而Linux系统中只有$\backslash n$(换行,0xA)。我现在主要使用的是Linux系统,但为了兼顾Windows系统,可能需要在读入配置文件后,先把其中的$\backslash r\backslash n$替换成$\backslash n$,然后才做词法分析。

目前暂时只解析裸键名,引号键名以后再考虑。


\subsection{10月6日}
在绘制状态转换图时,我们看到在识别某些内容时,可以按照不同的权衡有不同的处理方式。例如在判断整数时,可以在出现非数字符号就截止,也可以规定必须要出现空格、换行符或\#才截止,两种方式一个宽松,一个严格,各有各的好处与不足。前一种方式对TOML的书写格式比较宽松,但也因为过于宽松可能导致混乱,并增加后期处理的负担。后一种方式要求严格,书写时会有更多约束,但可以减少后期处理的工作量。这里说的后期处理主要是指语法分析阶段。


\subsubsection{10月7日}
随着状态转移图绘制的深入,会让人感到越来越繁琐,或许应该创建一个专门的工具来绘制,并且在绘制完成后自动转换成相应表格直接供词法分析器调用。这个工具的原理并不复杂,麻烦的是图形操作方面的支持问题,这将涉及到图形库方面,这是一个老话题了,先放一放。


\subsection{10月12日}
绘制状态转换图时,我曾经想到对于不合法的符号要如何在图中去处理,这是一种流程图的思维习惯。实际上,在状态转换图中并不需要显式指明如何处理不合法的符号,而是已经暗含了处理方式。合法的符号串可以从状态转换图的开始(start)处走到某一个终点,不合法的符号是没有路径的,在程序处理上会自动跳到错误处理模块,通常会向用户报告某行某列出现词法错误。通常情况下,每发现一个错误就退出程序并报告此错误,也可以把每一行视为一个单元,全部扫描后统一报告。全部扫描的方式还有一些细节问题需要考虑,并非简单的逐行处理就可以。

在把状态转换图映射为状态转换表时,需要把使用到的符号、状态都列出来,这项工作的繁琐程度会随着语言的复杂程度的增加而增加。对于本项目,即使只是简化版的,其状态转换图已经有些繁琐。


\section{2023年11月}


\subsection{11月13日}
在绘制状态转换图时,对于各个状态的编号,一开始我会习惯性的从1开始顺序编写。这样做对于后面的修改并不方便,因为每次修改中间的编号都要对后续的编号重新编写。因此,现在改为使用分段编号,例如注释是100开头,祼键使用200开头。这种方法就需要在状态转换表中增加一列参数来标明编号,以取代原来隐含的自然顺序编号。在实际的程序处理中,会多出一些用于判断编号的代码,虽然会增加一点开销,但有利于设计。

原本我打算在LibreOffice Draw中用一张图来完整描绘状态转换图,但发现这张图越来越大,以致于A0纸都放不下,因此将其拆分为各段,并使用LaTeX的tikz宏包来绘制。虽然也可以在LibreOffice Draw中分页绘制,但为了方便本记录的查阅,就还是放在LaTeX中吧。