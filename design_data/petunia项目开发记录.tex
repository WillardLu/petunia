\documentclass[oneside,fontset=founder]{ctexbook}

\usepackage{geometry}% 用于页面设置
\usepackage[dvipsnames, svgnames, x11names]{xcolor}% 颜色支持
\usepackage{graphicx}% 图形支持
\usepackage[
  colorlinks=true,
  linkcolor=Navy,
  urlcolor=Navy,
  citecolor=Navy,
  anchorcolor=Navy
]{hyperref}% 设置超链接颜色
\usepackage{enumerate}% 枚举支持
\usepackage{minted}% 代码显示支持

% 纸张设置
\geometry{
  b5paper,
  left = 1in,
  right = 1in,
  top = 1in,
  bottom = 1in
}

% 设置章节标题左对齐,+=表示在原有格式上追加,如果只有=则表示完全替换
\ctexset{
  chapter/format += \raggedright,
  section/format += \raggedright,
  subsection/format += \raggedright,
  subsubsection/format += \raggedright,
}

\emergencystretch = \maxdimen% 断字处理
\setlength{\parindent}{2em}% 缩进
\setlength{\parskip}{1ex} % 段间距

% 定义颜色
% \definecolor{pageYellow}{HTML}{F2F1D7}

% \pagecolor{pageYellow}


% ------------------ 开始 -------------------


\begin{document}


% ------------------ 封面 -------------------
\begin{titlepage}%
\begin{center}
  \quad

  \vspace{2ex}

  \includegraphics[width=.4\textwidth]{images/cover.png}

  \vspace{4ex}

  \Huge\heiti Petunia项目开发记录\normalsize\normalfont

  \vspace{4ex}

  陆巍
  \vfill%
  % Bottom of the page
  2023年10月5日
\end{center}
\end{titlepage}


% ------------------ 前言 -------------------
\frontmatter% 关闭章节序号,页码使用罗马数字


\chapter{前言}



% ------------------ 目录 -------------------
\tableofcontents% 生成目录


% ------------------ 正文 -------------------
\mainmatter


\chapter{开发日记}


\section{2023年10月}


\subsection{10月5日}
这个项目最初只是打算使用简单的判断方法来解决,但想到以后要开发其他编译器,所以先用这个微小项目来练练手。项目将按照编译器开发方法来实现,当然,本项目过于微小,大概只会用到词法分析与语法分析。

Windows系统和Linux系统在文本处理上是有些差异的,其中的换行就不相同。Windows系统中的换行实际上包含了两个字符,即$\backslash r$(回车,0xD)与$\backslash n$(换行,0xA),而Linux系统中只有$\backslash n$(换行,0xA)。我现在主要使用的是Linux系统,但为了兼顾Windows系统,可能需要在读入配置文件后,先把其中的$\backslash r\backslash n$替换成$\backslash n$,然后才做词法分析。

目前暂时只解析裸键名,引号键名以后再考虑。


\subsection{10月6日}
在绘制状态转换图时,我们看到在识别某些内容时,可以按照不同的权衡有不同的处理方式。例如在判断整数时,可以在出现非数字符号就截止,也可以规定必须要出现空格、换行符或\#才截止,两种方式一个宽松,一个严格,各有各的好处与不足。前一种方式对TOML的书写格式比较宽松,但也因为过于宽松可能导致混乱,并增加后期处理的负担。后一种方式要求严格,书写时会有更多约束,但可以减少后期处理的工作量。这里说的后期处理主要是指语法分析阶段。


\subsubsection{10月7日}
随着状态转移图绘制的深入,会让人感到越来越繁琐,或许应该创建一个专门的工具来绘制,并且在绘制完成后自动转换成相应表格直接供词法分析器调用。这个工具的原理并不复杂,麻烦的是图形操作方面的支持问题,这将涉及到图形库方面,这是一个老话题了,先放一放。


\subsection{10月12日}
绘制状态转换图时,我曾经想到对于不合法的符号要如何在图中去处理,这是一种流程图的思维习惯。实际上,在状态转换图中并不需要显式指明如何处理不合法的符号,而是已经暗含了处理方式。合法的符号串可以从状态转换图的开始(start)处走到某一个终点,不合法的符号是没有路径的,在程序处理上会自动跳到错误处理模块,通常会向用户报告某行某列出现词法错误。通常情况下,每发现一个错误就退出程序并报告此错误,也可以把每一行视为一个单元,全部扫描后统一报告。全部扫描的方式还有一些细节问题需要考虑,并非简单的逐行处理就可以。

在把状态转换图映射为状态转换表时,需要把使用到的符号、状态都列出来,这项工作的繁琐程度会随着语言的复杂程度的增加而增加。对于本项目,即使只是简化版的,其状态转换图已经有些繁琐。

\end{document}